\documentclass[letterpaper,11pt]{article}
% Packages
\usepackage[spanish]{babel}
\usepackage[utf8]{inputenc}
\usepackage[usenames,dvipsnames]{color}
\usepackage[margin=1in]{geometry}
\usepackage[T1]{fontenc}
\usepackage{lmodern}
\usepackage{textcomp}
\usepackage{hyperref}
\usepackage{amsmath}
\usepackage{amsthm}
\usepackage{fancyhdr}
\usepackage{titlesec}
\usepackage{enumitem}

\definecolor{blue(pigment)}{rgb}{0.2,0.2,0.6}

% Formatting
\hypersetup{colorlinks= true, linkcolor=blue(pigment), urlcolor=blue(pigment)}
\spanishdecimal{.}
\tolerance=1000
\hyphenpenalty=1000
\setlength{\parindent}{0in}
\setlength{\parskip}{0.1in}
\setlength\textheight{8.2in}
\setlength\topmargin{-1in}
\lhead{\sc Estatutos Centro de Alumnos de Matemáticas\\Pontificia Universidad Católica de Chile}
\chead{}
\rhead{}
\lfoot{}
\cfoot{\thepage}
\rfoot{}
\fancypagestyle{plain}{%
\fancyhf{}
\cfoot{\thepage}
\lfoot{}
\renewcommand{\headrulewidth}{0pt}}
\renewcommand{\headrulewidth}{1 pt}
\renewcommand{\footrulewidth}{0 pt}
\headheight=75pt

% Article Environment
\newcounter{art}
\newenvironment{art}{\textbf{
    Art.\refstepcounter{art} \theart .
}}{}

%\titleformat{\section}[display]{\scshape \Large}{Título \thesection:}{}{}[]

% Enumeraciones
\renewcommand{\theenumi}{\roman{enumi}}
\renewcommand{\labelenumi}{\theenumi.}
\renewcommand{\theenumii}{\arabic{enumii}}
\renewcommand{\labelenumii}{(\theenumii)}
\renewcommand{\theenumiii}{\roman{enumiii}}
\renewcommand{\labelenumiii}{\theenumiii.}
\renewcommand{\theenumiv}{(\alph{enumiv})}
\newcommand{\HRule}{\rule{\linewidth}{0.5mm}}
\newcommand{\aref}[1]{\hyperref[#1]{\ref*{#1}}}
\newcommand{\aaref}[2]{\hyperref[#2]{\ref*{#1}, letra \ref*{#2}}}
\makeatletter \renewcommand\p@enumii{\theenumi, } \makeatother
\makeatletter \renewcommand\p@enumiii{\theenumii, } \makeatother
\makeatletter \renewcommand\p@enumiv{\theenumiii, } \makeatother

% Formalidades respecto al título, autor y fecha de modificación
\title{Estatutos}
\author{Centro de Alumnos de Matemáticas}
\date{Marzo 2020}

% Inicio del documento
\begin{document}
\pagenumbering{gobble}
\thispagestyle{plain}
\vspace*{-75pt}

\begin{center}
    \begin{Large}
        {\bf
            ESTATUTOS DEL CENTRO DE ALUMNOS DE MATEMÁTICAS

            PONTIFICIA UNIVERSIDAD CATÓLICA DE CHILE
        }
    \end{Large}

    \vspace*{30pt}

\end{center}
\tableofcontents
\newpage
\pagenumbering{arabic}

\section{Declaración de Principios}\label{principios}
``Nosotros, la comunidad de estudiantes de matemática y estadística ratificamos nuestro compromiso con la defensa de los derechos e intereses de todos los estudiantes de la Facultad de Matemáticas, siendo rasgos fundamentales del centro de alumnos de matemáticas su naturaleza participativa, pluralista y el respeto a las decisiones de las mayorías. Junto con esto, manifestamos nuestro respeto a los valores católicos.''

\section{Normas Generales}\label{normasGenerales}
\begin{art}\label{}
    Se establecen como normas estatutarias del centro de alumnos de matemáticas (CAM) las contenidas en los presentes estatutos.
\end{art}

\begin{art}\label{}
    El CAM es el organismo que representa ante la comunidad universitaria de la Pontificia Universidad Católica de Chile (UC) a:
    \begin{enumerate}
        \item Estudiantes de pregrado regulares de la facultad de matemáticas.
        \item Estudiantes de postgrado regulares que manifiesten su deseo de ser representados.
    \end{enumerate}
    Llamaremos a ``estudiantes representados'' al conjunto descrito por los puntos anteriores.
\end{art}

\begin{art}\label{finalidadesCAM}
    El CAM tiene por finalidades:
    \begin{enumerate}
        \item Velar por los derechos e intereses de los estudiantes representados.
        \item Promover, impulsar y desarrollar las actividades conducentes a la correcta formación, integración y recreación de los estudiantes representados.
        \item Procurar que la enseñanza de la Facultad de Matemáticas (desde ahora ``la Facultad'') sea de excelencia, creando profesionales comprometidos con la sociedad.
        \item Administrar los fondos puestos a disposición del CAM por su antecesor, por la Facultad y por la FEUC. Asimismo, administrar los recursos que se generen mediante actividades, donaciones, entre otros.
    \end{enumerate}
\end{art}

\begin{art}\label{definicionesOrganismos}
    Los organismos directivos del CAM son los siguientes:
    \begin{enumerate}
        \item Directiva: Está conformada por cinco miembros. En orden jerárquico:
              \begin{enumerate}
                  \item Presidente
                  \item Vicepresidente Interno
                  \item Vicepresidente Externo
                  \item Secretario General
                  \item Tesorero
              \end{enumerate}
        \item Consejo Ejecutivo: Está conformado por miembros designados por el Presidente de la Directiva según lo que estime conveniente, especificando su labor, siendo aprobados individualmente por el Consejo Estudiantil.
        \item Consejo de Delegados: Se conformará por los delegados de cada generación de cada carrera, un delegado de Postgrado y un delegado correspondiente a todos los estudiantes no considerados en los grupos anteriores independiente de su carrera. Hay cinco generaciones de Estadística y cuatro de Matemáticas, se dividen como se ve en la tabla.
              \begin{center}
                  \begin{tabular}{|c|c|c|}
                      \hline
                      Año ingreso   & Generación Matemáticas & Generación Estadística \\
                      \hline
                      Año actual    & Generación 1           & Generación 1           \\
                      Año actual -1 & Generación 2           & Generación 2           \\
                      Año actual -2 & Generación 3           & Generación 3           \\
                      Año actual -3 & Generación 4           & Generación 4           \\
                      Año actual -4 &                        & Generación 5           \\
                      \hline
                  \end{tabular}
              \end{center}
        \item Consejerías Académicas: Hay dos consejerías académicas, cada una corresponde a una carrera de pregrado, y cada una está conformada por un Consejero y además por un Subconsejero si así el Consejero lo desea.
        \item Consejo Estudiantil: Está conformado por el Consejo de Delegados, las Consejerías Académicas, el Consejero Territorial y dos miembros de la Directiva. Los Consejos que se convoquen serán abiertos y podrá participar con derecho a voz todo estudiante representado que así lo desee.
    \end{enumerate}
\end{art}

\begin{art}\label{}
    El TRICEL es un organismo no directivo del CAM. Y está conformado como se describe en la sección \ref{TRICEL}.
\end{art}

\begin{art}\label{}
    Los estudiantes que estén en el proceso de cambio de carrera serán representados por el CAM hasta que el cambio se haga efectivo. Además, al hacerse efectivo el cambio, el estudiante deberá informar esto a los respectivos centros de estudiantes.
\end{art}

\section{De las Atribuciones y Deberes}\label{funcionesAtribuciones}

\begin{art}\label{atribucionesEstudiantes}
    Son atribuciones de los estudiantes representados:
    \begin{enumerate}
        \item Manifestar libremente su opinión en las Asambleas Generales\footnote{Como se definen en el artículo \ref{asambleas}} y otros espacios destinados por el CAM para tal efecto.
        \item Publicar información y/u opiniones en espacios destinados por el CAM, siempre y cuando el CAM no ponga objeciones.
        \item Participar en las actividades que organice el CAM.
        \item Exigir información sobre el trabajo del CAM en forma oral o escrita.
        \item Proponer proyectos al CAM.
        \item Solicitar espacios, equipos y/o materiales que sean del CAM para realizar actividades que brinden beneficios a la comunidad universitaria.
    \end{enumerate}
\end{art}

\begin{art}\label{deberesCAM}
    Son deberes de los miembros del CAM:
    \begin{enumerate}
        \item Informar a cada uno de los estudiantes de sus derechos.
        \item Hacer efectivo su derecho a voto en las elecciones.
        \item Asistir a las asambleas generales al ser convocadas.
        \item Respetar la opinión y los espacios de democracia que se generen.
        \item Velar por el cumplimiento de los estatutos.
    \end{enumerate}
\end{art}

\begin{art}\label{funcionesDirectiva}
    Son atribuciones y deberes de la Directiva:
    \begin{enumerate}
        \item Desarrollar el plan de trabajo presentado y dar cuenta trimestral en las Asambleas Generales\footnotemark[2]. %Ver para que haga referencia a la otra nota al pie.
        \item La Directiva por tanto tendrá que convocar a lo menos dos asambleas por semestre.
        \item Crear y promover el desarrollo de actividades de orden estudiantil y/o de formación general e integral en el estudiantado.
        \item Apoyar a las vocalías.
        \item Pronunciarse ante los problemas estudiantiles y sociales, e informar su postura al respecto. % ? Revisar de nuevo
        \item Representar ante autoridades universitarias, organismos superiores y la Federación de Estudiantes (FEUC), a los estudiantes representados.
        \item Administrar los bienes que estén a disposición del CAM.
        \item Convocar a sesiones ordinarias y extraordinarias de la Asamblea General\footnote{Como se describe en el artículo \ref{asambleas}}.
        \item Informar a los estudiantes de las actividades realizadas y a realizar de posible interés del estudiantado.
        \item Ningún miembro de la Directiva podrá abstenerse en una votación de Asamblea General.
        \item Asistir a los Consejos de Federación, de Facultad y Estudiantil según corresponda.
        \item En cuanto a la toma de decisiones, la Directiva tendrá la facultad de sesionar cuantas veces estime conveniente para cumplir con las funciones que contemple el actual estatuto.
        \item Las decisiones serán tomadas luego del acuerdo de la totalidad de la Directiva, o como los miembros de esta estimen conveniente.
        \item Todas las demás disposiciones que el presente estatuto estipule.
    \end{enumerate}
\end{art}

\begin{art}\label{funcionesPresidente}
    Son funciones del Presidente:
    \begin{enumerate}
        \item Tomar la representación del CAM tanto dentro como fuera de la universidad, de acuerdo a las finalidades del CAM.
        \item Garantizar la existencia de un vocero de las inquietudes de los estudiantes.
        \item Asegurar una estructura eficiente en las Asambleas Generales.
        \item Presidir reuniones de Directiva.
        \item Realizar al menos dos cuentas públicas cada semestre, una al principio y otra al final del semestre, donde se deberá entregar una evaluación con los hitos más importantes del semestre, junto con un estado financiero que refleje el gasto de dinero del CAM.
        \item Ampliar el número del Consejo Ejecutivo, sus funciones y atribuciones dentro de lo que estime conveniente, pero bajo la aprobación del Consejo Estudiantil.
    \end{enumerate}
\end{art}

\begin{art}\label{funcionesVicepresidenteInterno}
    Son funciones del Vicepresidente Interno:
    \begin{enumerate}
        \item Apoyar al Presidente en las funciones que este deba cumplir con la comunidad de la Facultad.
        \item Subrogar al Presidente en su ausencia.
        \item Velar por el cumplimiento de las funciones de los distintos miembros de la Directiva y de los miembros del Consejo Estudiantil.
        \item Mediar las relaciones entre la Directiva y las autoridades, profesores y funcionarios de la Facultad. Asimismo, deberá coordinar cualquier trabajo con el Consejo Estudiantil.
        \item Difundir toda la información que la Directiva deba y/o quiera difundir por todos los medios oficiales.
    \end{enumerate}
\end{art}

\begin{art}\label{funcionesVicepresidenteExterno}
    Son funciones del Vicepresidente Externo:
    \begin{enumerate}
        \item Apoyar al Presidente en las funciones que este deba cumplir con la comunidad universitaria.
        \item Subrogar al Vicepresidente Interno en caso de ausencia de este.
        \item Mediar las relaciones entre la Directiva y las autoridades universitarias, la Federación, Movimientos Políticos, las instituciones y organismos no pertenecientes a la universidad.
        \item Informar e invitar al Consejero Territorial a los consejos estudiantiles cuando estos se realicen.
    \end{enumerate}
\end{art}

\begin{art}\label{funcionesSecretario}
    Son funciones del Secretario General:
    \begin{enumerate}
        \item Publicar en todos los medios oficiales las citaciones a Asambleas Generales\footnote{Como se definen en artículo \ref{asambleas}}, además de publicar las actas de estas sesiones.
        \item Mantener el inventario al día y tenerlo a disposición de la Directiva cuando esta lo estime conveniente.
        \item A comienzos de año dar a conocer los estatutos que rigen al CAM, publicándolos en todos los medios oficiales.
        \item Recibir proyectos de reglamentos y de reformas a los estatutos.
        \item Tomar acta de las Asambleas y reuniones de Directiva tanto ordinarias como extraordinarias.
        \item Publicar las actas de asambleas en todos los medios oficiales.
    \end{enumerate}
\end{art}

\begin{art}\label{funcionesTesorero}
    Son funciones del Tesorero:
    \begin{enumerate}
        \item Llevar la contabilidad del CAM, entendiéndose como mantener el libro de ingresos y egresos al día, mantener todas la boletas del CAM y todo lo que compruebe las cuentas realizadas. Teniéndolos a disposición de la Directiva y miembros del Consejo Estudiantil cuando este lo estime conveniente.
        \item Presentar un informe trimestral de haberes.
        \item Administrar la cuenta de ahorros y/o corriente del CAM, si existiere.
        \item Administrar los fondos de becas del CAM.
        \item Evaluar y administrar los gastos y ganancias que impliquen los proyectos que se generen de parte de la comunidad.
    \end{enumerate}
\end{art}

\begin{art}\label{funcionesDelegados}
    Son funciones de los Delegados:
    \begin{enumerate}
        \item Ayudar a la Directiva en las actividades que se realicen para la comunidad, tanto en la realización como en su difusión.
        \item Difundir información que el CAM entregue al nivel que pertenezca.
        \item Asistir a los Consejos Estudiantiles que se convoquen.
    \end{enumerate}
\end{art}

\begin{art}\label{atribucionesDelegados}
    Son atribuciones de los Delegados:
    \begin{enumerate}
        \item Proponer y llevar a cabo proyectos aprobados por el Consejo Estudiantil.
    \end{enumerate}
\end{art}

\begin{art}\label{funcionesConsejeriaAcademica}
    Son funciones y atribuciones de la Consejería Académica:
    \begin{enumerate}
        \item Ser defensor académico, y todo lo que ello conlleva, de los estudiantes de pregrado que decidan ser representados.
        \item Asistir a los comités curriculares de la Facultad, según lo establecido en la Normativa de Comités Curriculares de la Vicerrectoría Académica.
        \item Asistir a los Consejos Académicos.
        \item Estudiar y desarrollar proyectos y propuestas académicas para el alumnado. También organizar actividades dirigidas al bienestar de los estudiantes representados, y mantener constante contacto con estos.
        \item Estar al tanto e informar al alumnado sobre el proceso de permanencia.
        \item Representar a los estudiantes representados en el resto de los espacios pertinentes al cargo según estime la contingencia para la correcta gestión de este.
    \end{enumerate}
\end{art}

\begin{art}\label{funcionesConsejeroTerritorial}
    Son funciones y atribuciones del Consejero Territorial que represente a la Facultad de Matemática todos aquellos dispuestos en los estatutos FEUC, y aquellos dispuestos en este estatuto.
\end{art}

\begin{art}\label{funcionesConsejoEstudiantil}
    Son funciones y atribuciones del Consejo Estudiantil:
    \begin{enumerate}
        \item Elegir un director de Consejo entre los integrantes que conformen dicho Consejo, quien moderará el Consejo.
        \item Elegir un secretario entre los integrantes que conformen dicho Consejo, quien llevará actas del Consejo para su posterior registro. Las actas quedarán a libre disposición del estudiantado.
        \item Velar por el buen trabajo del CAM.
        \item Proponer y modificar proyectos.
        \item Votar sobre proyectos que el CAM proponga.
        \item En caso de que el Consejo lo estime necesario, destituir a cualquier miembro de un organismo directivo del CAM. Esto se regirá según lo descrito en el artículo \ref{perdidaCargosCAM} y el artículo \ref{destitucionesConsejo} de la sección \ref{vacancias}.
    \end{enumerate}
\end{art}

\section{De las Asambleas y Consejos}\label{asambleasConsejos}
\begin{art}\label{asambleas}
    Las Asambleas tienen carácter informativo y/o consultivo, pudiendo transformarse en resolutivas si es aprobado por la mayoría simple de los presentes en la Asamblea. En el caso de que estas sean resolutivas, deberán contar con una asistencia mínima del 35\% de los estudiantes representados. Hay dos tipos de Asambleas:
    \begin{enumerate}
        \item Asamblea Ordinaria: Debe ser convocada al menos dos veces cada semestre, la cual deberá ser citada con al menos tres días hábiles de anticipación.
        \item Asamblea Extraordinaria: Convocada por la Directiva, o por un grupo de mínimo 7 estudiantes representados que previamente haya entregado sus firmas a la Directiva. Esta deberá ser citada con al menos dos días hábiles de anticipación.
    \end{enumerate}
\end{art}

\begin{art}\label{consejos}
    Los Consejos Estudiantiles son de dos tipos, Consejo Ordinario y Consejo Extraordinario. Cuando estas sean convocadas y no esté presente el Director o el Secretario del Consejo se designará a un Director o un Secretario interino, según corresponda.
    \begin{enumerate}
        \item Consejo Ordinario: Es convocada por el director del Consejo como mínimo una vez cada mes, exceptuando los meses de Enero, Febrero y Julio, y con al menos dos días hábiles de anticipación, cualquier decisión resolutiva requiere un mínimo del 50\% de la asistencia del total de los integrantes del Consejo Estudiantil.
        \item Consejo Extraordinario: Puede ser convocado por cualquier miembro del CAM, con al menos tres horas de anticipación, cualquier decisión resolutiva require un mínimo del 60\% de la asistencia del total de los integrantes del Consejo Estudiantil.
    \end{enumerate}
\end{art}

\section{De la elección de cargos}\label{elecciones}

\begin{art}\label{eleccionesRequerimientosPostulación}
    Para postular a un cargo del CAM, se requiere no estar en algunas de las alternativas mencionadas en el artículo \ref{perdidaCargosCAM}.
\end{art}


\begin{art}\label{eleccionesConvocación}
    Será responsabilidad de la Directiva en ejercicio convocar a las elecciones correspondientes.
\end{art}

\subsection{De la elección de la Directiva}\label{eleccionesCAM}
\begin{art}\label{}
    La Directiva CAM se elegirá anualmente por medio de una votación como se describe en la sección \ref{votaciones}.
\end{art}

\begin{art}\label{eleccionesCAMPostulacion}
    Cada postulación tendrá que designar a un integrante por cada posición de la Directiva. Además, para formalizar la inscripción se debe hacer llegar la lista con todos los integrantes a la Directiva en ejercicio.
\end{art}

\begin{art}\label{eleccionesCAMFechas}
    Las elecciones se harán a lo menos cinco días hábiles y a los más ocho días hábiles después del término del proceso de inscripción. Este último se deberá realizar durante la segunda semana de Octubre y debe durar cinco días hábiles. La fecha exacta del proceso de inscripción será determinada por la Directiva en ejercicio, con a lo menos cinco días hábiles de antelación.
\end{art}

\begin{art}\label{eleccionesCAMPublicacion}
    Al término del proceso de inscripción el Secretario General deberá publicar en los medios oficiales las listas de candidatos y sus respectivos programas.
\end{art}

\begin{art}\label{eleccionesCAMInscripcion}
    Si una vez cumplido el plazo de inscripción no hay postulaciones, el Consejo Estudiantil decidirá si agregará un plazo extraordinario de a lo más de cinco días hábiles o designar una Directiva interina.
\end{art}

\begin{art}\label{eleccionesCAMFalla}
    En caso de no tener postulaciones o de ser rechazadas las listas, la Directiva interina compuesta por miembros del Consejo Estudiantil conformará una Directiva hasta una nueva convocatoria el siguiente año, junto con la elección de delegados.
\end{art}

\begin{art}\label{eleccionesCAMCandidatos}
    Sólo podrán ser candidatos a los cargos de la Directiva aquellos estudiantes de pregrado que sean integrantes de la Facultad de Matemáticas, y que no se encuentren en proceso de egreso.
\end{art}

\begin{art}\label{eleccionesCAMVotacion}
    Para que la elección tenga validez se debe cumplir con los puntos descritos en la sección \ref{votaciones}.
\end{art}

\subsection{De la elección de la Consejería Académica}\label{eleccionesConsejeria}

\begin{art}\label{eleccionesConsejeriaLista}
    Cada postulación a la Consejería Académica será conformada por un Consejero Académico.
\end{art}

\begin{art}\label{eleccionesConsejeriaFalla}
    Si al final del plazo de inscripción no hay postulaciones, se agregará un plazo extraordinario de a lo más cinco días hábiles. En caso de no haber una postulación en este plazo extraordinario, se considerará el puesto vacante y se seguirá lo descrito en el artículo \ref{vacanciasConsejero}.
\end{art}

\begin{art}\label{eleccionesConsejeriaVotacion}
    La elección se realizará en conjunto con la de la Directiva, por lo cual deberán seguir la condiciones establecidas en la sección \ref{votaciones}.
\end{art}

\subsection{De la elección de los Delegados}\label{eleccionesDelegados}

\begin{art}\label{eleccionesDelegadosFecha}
    La elección de Delegados será la segunda semana de abril, la fecha la dispondrá la Directiva al mando y no durará más de cinco días hábiles.
\end{art}

\begin{art}\label{eleccionesDelegadosCandidatos}
    Todo estudiante representado podrá postularse a ser Delegado de su generación. Las votaciones se llevarán a cabo como se describe en la subsección \ref{votacionesDelegados}.
\end{art}

\begin{art}\label{eleccionesDelegadosFalla}
    En caso de que en alguna generación no exista una postulación, el Secretario General deberá suplir las funciones del Delegado correspondiente hasta que se consiga algún representante.
\end{art}

\subsection{De la elección del TRICEL}

\begin{art}\label{eleccionesTRICELFecha}
    La elección del TRICEL será la segunda semana de abril, junto con la elección de delegados.
\end{art}

\begin{art}\label{eleccionesTRICELCandidatos}
    Todo estudiante representado podrá postularse a ser miembro del TRICEL. Las votaciones se llevarán acabo como se describe en la subsección \ref{votacionesTRICEL}
\end{art}

\section{Del Tribunal Calificador de Elecciones (TRICEL)}\label{TRICEL}

\begin{art}\label{TRICELMision}
    El TRICEL es el organismo encargado de organizar, vigilar y concretar todo proceso de votación a nivel CAM.
\end{art}

\begin{art}\label{TRICELDescripcion}
    El TRICEL estará compuesto por un mínimo de cuatro estudiantes representados, los cuales no tendrán algún otro cargo dentro del CAM.
\end{art}

\begin{art}\label{TRICELFunciones}
    Serán funciones del TRICEL:
    \begin{enumerate}
        \item Velar por la realización y garantizar la transparencia del proceso de votaciones.
        \item Conocer cualquier asunto relacionado con las votaciones que fiscalice.
        \item Calificar las votaciones dando su dictamen respecto de la legitimidad o nulidad, de naturaleza parcial o total, del proceso.
        \item Determinar la lista de personas con derecho a voto de acuerdo a lo estipulado en el presente estatuto.
        \item Atender, investigar y resolver los reclamos y observaciones presentadas con respecto al proceso de votación por cualquier estudiante representado.
        \item Determinar, publicar, solicitar y distribuir el material necesario para la implementación de las votaciones.
    \end{enumerate}
\end{art}

\begin{art}\label{TRICELResolucion}
    Las resoluciones del TRICEL serán tomadas por acuerdo unánime y solo serán apelables ante el mismo tribunal por vía de reconsideración.
\end{art}

\section{Destituciones, Renuncias y Vacancias}\label{vacancias}

\begin{art}\label{perdidaCargosCAM}
    Todo cargo del CAM se mantiene hasta que se designe una nueva persona en el cargo correspondiente, y se pierde si se cumple alguna de las siguientes alternativas:
    \begin{enumerate}
        \item Por renuncia voluntaria.
        \item Por destitución por medios definidos en la sección \ref{destituciones}
        \item Por expulsión de la Universidad.
        \item Por abandono de la carrera.
        \item Por suspensión de la carrera de forma voluntaria y temporal.
        \item Al finalizar sus estudios de Pregrado en la Facultad, con excepción del delegado de Postgrado.
        \item Al terminar el semestre con promedio inferior a 4,0.
        \item Al tener dos o más alertas académicas (sistema de permanencia).
    \end{enumerate}
\end{art}

\subsection{Destituciones}\label{destituciones}

\begin{art}\label{destitucionesConsejo}
    El Consejo Estudiantil puede destituir miembros del CAM, con excepción del Consejero Territorial, si es que este así lo decide por una votación interna no secreta. Esto será comunicado por todos los medios oficiales del CAM, e incluirá las razones de esta destitución y la fecha de la misma.
\end{art}

\begin{art}\label{destitucionesAsamblea}
    En una Asamblea General se puede proponer la moción de destituir a algún representante del CAM, esta moción se tiene que aprobar bajo lo establecido en la sección \ref{votacionesGeneral}.
\end{art}


\subsection{Vacancias y Renuncias}\label{vacanciasRenuncias}

\begin{art}\label{renunciasGeneral}
    Cada integrante del CAM tiene la facultad de renunciar en caso de que así lo desee, pero esta facultad no puede ser usada de forma simultánea por más de un miembro, excepto en caso de renuncia de la Directiva completa.
\end{art}

\begin{art}\label{renunciasDirectivaMiembro}
    Al renunciar un integrante de la Directiva, esta tiene la facultad de proponer un posible reemplazante, el cual tendrá que ser aprobado por el Consejo Estudiantil.
\end{art}

\begin{art}\label{vacanciasDirectivaMiembro}
    En caso de que un integrante de la Directiva no proponga un reemplazante, o que el reemplazo no sea aprobado, este puesto será reemplazado de la siguiente forma:
    \begin{enumerate}
        \item Si el cargo es ejecutivo (i.e. Presidente, Vice-Presidente Interno, Vice-Presidente Externo) este será reemplazado por el siguiente cargo ejecutivo en la línea de sucesión, la cual por defecto será, en orden, Presidente, Vice-Presidente Interno y por último Vice-Presidente Externo.
        \item Si el cargo es administrativo (i.e. Secretario, Tesorero) este será reemplazado por el siguiente cargo administrativo en la línea de sucesión, la cual por defecto será, en orden, Secretario y por último Tesorero.
        \item En caso de que el último cargo de alguna de las líneas de sucesión quede vacío se sigue en orden alfabético los puntos descritos a continuación. Además, en todos los casos debe ser aprobado por el Consejo Estudiantil con una votación de mayoría simple:
        \begin{enumerate}[label=(\alph*)]
            \item La Directiva puede proponer un miembro del Consejo Ejecutivo.
            \item La Directiva puede proponer un miembro del Consejo Estudiantil.
            \item El Consejo Estudiantil puede proponer un miembro del Consejo Ejecutivo.
            \item El Consejo Estudiantil puede proponer un miembro del Consejo Estudiantil.
        \end{enumerate}
    \end{enumerate}
\end{art}

\begin{art}\label{vacanciasDirectivaPlazos}
    En caso de quedar vacante la posición de un integrante de la Directiva, el integrante en cuestión tiene que presentar su renuncia con una semana de anticipación, excepto en el caso de la tesorería donde esta necesitará un plazo extra de dos semanas, para facilitar el proceso de cambio administrativo.
\end{art}

\begin{art}\label{vacanciasDirectivaCompleta}
    En caso de quedar vacante la posición de toda la Directiva asumirá el Consejo Estudiantil en forma de una Directiva interina, y se llamará a elecciones inmediatamente después de la renuncia. En caso de que la renuncia sea el primer semestre, la Directiva electa terminará el período en curso, en caso contrario, la Directiva electa asume el periodo en curso y además un período completo.
\end{art}

\begin{art}\label{vacanciasConsejero}
    En caso de quedar vacante la posición del consejero académico, el cargo lo asume el subconsejero académico, y en caso de la ausencia de éste, lo asume el vicepresidente interno, quedando con ambos cargos.
\end{art}

\begin{art}\label{vacanciasConsejoDelegados}
    En caso de quedar vacante la posición de un integrante del Consejo de Delegados, se llamará a elecciones abiertas bajo las mismas reglas expuestas en el artículo \ref{votacionesDelegados}, con la siguiente excepción: si no hay candidato, entonces el Consejo Estudiantil designará por votación interna a uno de sus miembros para que asuma este cargo; luego, se hará una votación aprobatoria según lo establecido en el artículo \ref{votacionesDelegados}. En caso de que el candidato no sea aprobado, o que no se haya llegado a un acuerdo en el Consejo Estudiantil, el puesto queda vacío.
\end{art}

\begin{art}\label{vacanciasConsejoEstudiantil}
    En caso de quedar vacante la posición del presidente, o el secretario, del Consejo Estudiantil se volverán a elegir estos cargos después de finalizar los procesos de los artículos \ref{renunciasDirectivaMiembro}, \ref{vacanciasDirectivaMiembro}, \ref{vacanciasDirectivaCompleta}, \ref{vacanciasConsejero} y \ref{vacanciasConsejoDelegados}.
\end{art}

\section{Votaciones}\label{votaciones}

\subsection{Votaciones Generales}\label{votacionesGeneral}

\begin{art}\label{}
    Toda votación tiene un universo de votantes, el cual por defecto será los estudiantes representados.
\end{art}

\begin{art}\label{}
    Toda votación tiene un quórum mínimo para ser válida, el cual por defecto será de 40\%.
\end{art}

\begin{art}\label{}
    Toda votación tiene duración, la cual por defecto será dos días hábiles.
\end{art}

\begin{art}\label{}
    Toda votación para una elección tiene un sistema electoral para determinar los individuos que son elegidos en los cargos correspondientes. Por defecto, el sistema es de mayoría simple.
\end{art}

\begin{art}\label{}
    Toda votación para una elección tiene un mínimo de aprobación necesaria para que los candidatos puedan llegar a asumir el cargo correspondiente. Por defecto, este mínimo será 30\%.
\end{art}

\begin{art}\label{}
    Toda votación para una elección donde haya una postulación única, será de apruebo o rechazo por defecto.
\end{art}

\begin{art}\label{}
    Toda votación será fiscalizada y validada por el presente TRICEL.
\end{art}

\subsection{Votaciones Directiva}

\begin{art}\label{}
    La votación para la elección de la Directiva tiene como mínimo de aprobación necesaria de 40\%.
\end{art}

\subsection{Votaciones Consejería Académica}

\begin{art}\label{}
    En la votación para la elección de la Consejería Académica se tienen dos universos de votantes separados, compuestos por estudiantes representados de la carrera de Matemáticas y de la carrera de Estadística correspondientemente. Para cada universo de votantes se harán votaciones, las cuales serán llevadas a cabo en paralelo.
\end{art}

\subsection{Votaciones Delegados}\label{votacionesDelegados}

\begin{art}\label{}
    En la votación para la elección de Delegados de cada generación, de Postgrado y de los estudiantes representados no considerados en los grupos anteriores (ver artículo \ref{definicionesOrganismos}, iii) son universos de votantes distintos y separados, por ende para cada universo de votantes se harán votaciones, las cuales serán llevadas a cabo en paralelo.
\end{art}

\begin{art}\label{}
    La votación para la elección de Delegados tiene como mínimo de aprobación necesaria de 40\%.
\end{art}

\subsection{Votaciones TRICEL}\label{votacionesTRICEL}

\begin{art}\label{}
    El sistema electoral de la votación para la elección del TRICEL se describe a continuación: Se realiza una votación distinta para cada candidato de apruebo o rechazo, después se ordenan los candidatos por orden de aprobación y se eligen los cuatro candidatos válidos con la mayor aprobación.
\end{art}

\subsection{Del Plebiscito}\label{plebiscito}

\begin{art}\label{plebiscitoDescripcion}
    El plebiscito es la consulta directa a todos los estudiantes, realizada mediante una votación, sobre materias especificas y que tiene carácter vinculante al interior del CAM.
\end{art}

\begin{art}\label{plebiscitoConvocar}
    Pueden convocar a Plebiscito:
    \begin{enumerate}
        \item La Directiva por unanimidad.
        \item La Asamblea General con mayoría absoluta de sus votos.
        \item El Consejo Estudiantil con mayoría absoluta de sus votos.
    \end{enumerate}
\end{art}

\begin{art}\label{plebiscitoDocumento}
    En caso de que la Asamblea General convoque un plebiscito, se creará una comisión la cual estará encargada de crear un documento con las alternativas de forma clara, precisa y completa. En caso de que un organismo del CAM convoque un plebiscito, este tomará el lugar de la comisión.
\end{art}

\begin{art}\label{plebiscitoAntelacion}
    La convocatoria a plebiscito, junto al documento con las alternativas, deberá ser presentada con al menos cinco días hábiles de antelación al plebiscito.
\end{art}

\begin{art}\label{}
    En caso de que el plebiscito no sea válido, este no tendrá carácter vinculante.
\end{art}

\begin{art}\label{}
    El plebiscito estará a cargo de la Directiva, y el TRICEL ofrecerá encargados de mesa, los cuales serán testigos en los recuentos parciales y totales.
\end{art}

\section{Vocalías}\label{vocalias}
\begin{art}\label{}
    Las vocalías son organismos independientes con capacidad ejecutiva apoyadas por la Directiva.
\end{art}

\begin{art}\label{}
    Cada vocalía deberá estar conformada por un mínimo de dos estudiantes representados y deberá ser aprobada por el Consejo Estudiantil.
\end{art}

\begin{art}\label{}
    Como organismo independiente, estas pueden conseguir financiamiento propio y/o solicitar apoyo financiero formalmente a la Directiva, llámese, tiene  que haber una postulación escrita con un proyecto asociado; la cantidad de apoyo necesario y las formas en las que se usará este apoyo; y las firmas de todos los integrantes. El apoyo solicitado tiene que ser aprobado por el Consejo Estudiantil y la Directiva.
\end{art}


\section{De los Estatutos}\label{estatutos}% ? Mover, agregar y/o reescribir cosas de la sección
% TODO Rehacer la parte de la presente sección, bajo lo mencionado por el grupo de Matías Bruna. Ver las propuestas que se generen el viernes 22 de Mayo

\begin{art}\label{tiporeformas}
	Existen dos tipos de propuestas a reforma:
	\begin{enumerate}
		\item Propuesta de cambio gramatical.
		\item Propuesta de cambio estructural.
	\end{enumerate}
\end{art}

\begin{art}\label{cambiogramatical}
	Todo cambio gramatical debe ser discutido y aprobado en el Consejo Estudiantil.
\end{art}


\begin{art}\label{}
	Podrán llamar a cambio estructural de los presentes estatutos:
	\begin{enumerate}
		\item La Asamblea General, con mayoria simple.
		\item El Consejo Estudiantil, con dos tercios de los votos.
		\item Miembros de la Directiva, por unanimidad.
	\end{enumerate}
\end{art}

\begin{art}
Este llamado a cambio debe ser presentado en la Asamblea General, acompañado de argumentación que justifique la solicitud. Junto a esto, se debe hacer un llamado abierto, por todos los medios oficiales del CAM, para conformar una comisión. Esta realizará una propuesta de reforma a los presentes estatutos, la cual debe seguir los lineamientos del llamado a cambio antes mencionado.
\end{art}

\begin{art}\label{estatutosReformaAprobacion}
    Una vez la comisión antes mencionada tenga lista una propuesta a reforma, esta se deberá  presentar por escrito a la Asamblea General, explicando los cambios que se proponen, junto con su argumentación respectiva. Posteriormente, esta se llevará a votación siguiendo lo detallado en la sección \ref{votaciones}.
\end{art}

\begin{art}\label{vigenciareformas}
Toda reforma aprobada entrará en vigencia una vez publicada.
\end{art}

\begin{art}\label{articulostransitorios}
Si una reforma genera una incompatibilidad con los estatutos vigentes, esta deberá contener artículos transitorios que establezcan soluciones a estas incompatibilidades. La vigencia de estos no puede extenderse más de un año de su publicación y, terminado este plazo, automáticamente son eliminados de los estatutos vigentes.
\end{art}


\begin{art}\label{publicaciónreformas}
	El Secretario General tendrá la responsabilidad de publicar los cambios respectivos a más tardar tres días hábiles una vez aprobada la reforma.
\end{art}


% TODO Considerar convenciones en la documentación y los cambios de los estatutos para facilitar las futuras reformas y la documentación de los cambios.
\begin{art}\label{}
    Se deberá mantener un historial de cambios y propuestas, tanto aprobadas como rechazadas, de los estatutos en la plataforma GitHub (\url{https://github.com/CAM-UC/Estatutos}). La administración de este historial será responsabilidad del Secretario General, o de un estudiante aprobado por el Consejo Estudiantil, el cual deberá usar el sistema de “pull requests” y “branches” para mantener de manera ordenada los cambios realizados y propuestos, detallándolos de la mejor manera posible.
\end{art}

\begin{art}\label{}
    En caso de existir cualquier duda sobre la interpretación del presente estatuto, el Consejo Estudiantil deberá presentar una interpretación. Esta será discutida y podrá ser modificada en la Asamblea General, para posteriormente ser sometida a votación en la misma Asamblea.
\end{art}

% * La siguiente parte corresponde a la documentación de futuras y la presente reforma.
% * Todo equipo correspondiente a una reforma deberá añadirse abajo incluyendo la información pertinente (e.g. mes, año, miembros del equipo en cuestión)

\newpage
\textsc{Elaborado por Comisión de reforma de estatutos:}
Elías Alvear, Sebastián Avendaño, Matías Bruna, María Catalina Cárdenas, Ángela Flores, Nicholas Mc-Donnell, Romina Mercado, José Antonio Montenegro, Javier Reyes, Juan Pablo Vega y Paulina Vega.
\begin{sloppypar}
Facultad de Matemáticas, Pontificia Universidad Católica de Chile\\
Santiago, Mayo de 2020
\end{sloppypar}
\end{document}
